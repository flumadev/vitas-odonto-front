\documentclass[12pt,a4paper]{article}

\usepackage[T1]{fontenc}
\usepackage[utf8]{inputenc}
\usepackage[brazil]{babel}
\usepackage{geometry}
\usepackage{longtable}
\usepackage{array}
\usepackage{hyperref}
\usepackage{enumitem}
\usepackage{booktabs}
\usepackage{graphicx}
\usepackage{tikz}
\usetikzlibrary{arrows.meta,positioning,shapes.geometric}

\geometry{margin=2.5cm}
\setlist[itemize]{leftmargin=1.2cm}
\setlist[enumerate]{leftmargin=1.2cm}
\hypersetup{colorlinks=true,linkcolor=black,urlcolor=blue}

\input{../docs/requirements/mermaid-macros.tex}

\title{Documento de Análise e Levantamento de Requisitos\\Vitas Odonto}
\author{Produto, Engenharia e Operações}
\date{\today}

\begin{document}
\maketitle
\tableofcontents
\newpage

\section{Objetivo}
Congelar o escopo funcional e não funcional da versão 1.0 do módulo Odonto para entrada em \texttt{Ready for Build}, garantindo rastreabilidade entre dor operacional, requisitos e critérios de aceite.

\section{Escopo do Produto}
\subsection{Escopo funcional}
\begin{itemize}
  \item Agenda odontológica com autoagendamento e validação de disponibilidade.
  \item Confirmação e lembrete automáticos por WhatsApp/SMS.
  \item Prontuário odontológico digital com assinaturas eletrônicas.
  \item Fechamento de atendimento com status administrativo e histórico mínimo.
  \item Controle de funcionalidades por plano e perfil.
\end{itemize}

\subsection{Fora de escopo}
\begin{itemize}
  \item Teleconsulta em vídeo.
  \item Emissão fiscal completa (NFS-e).
  \item Crédito/financiamento de tratamento.
  \item Integrações com maquininha de cartão.
\end{itemize}

\section{Planos, Pacotes ou Modalidades}
\subsection{Planos previstos}
\begin{itemize}
  \item Consultório — agenda, prontuário e fechamento básico de atendimento.
  \item Clínicas — inclui automações de confirmação e indicadores operacionais.
  \item Rede — inclui governança multiunidade e políticas avançadas.
\end{itemize}

\subsection{Diretrizes por plano}
\begin{itemize}
  \item Regras de restrição devem ser aplicadas no backend.
  \item Interface não deve ser o único mecanismo de bloqueio.
\end{itemize}

\section{Perfis e Stakeholders}
\begin{itemize}
  \item Paciente — realiza agendamento e confirma presença.
  \item Recepção — opera agenda e acompanha respostas de confirmação.
  \item Dentista — registra evolução e documentos clínicos.
  \item Administrativo — acompanha fechamento de atendimentos e indicadores operacionais.
  \item Gestor — monitora faltas, ocupação e eficiência operacional.
\end{itemize}

\section{Premissas e Restrições}
\begin{itemize}
  \item A agenda deve respeitar regras de disponibilidade por profissional e tipo de procedimento.
  \item O fechamento de atendimento não depende de integrações financeiras para concluir o fluxo clínico.
  \item Mensageria depende de provedor externo com SLA contratual.
  \item Dados clínicos devem obedecer segregação por clínica e LGPD.
\end{itemize}

\section{Arquitetura e Tecnologias Sugeridas}
\renewcommand{\arraystretch}{1.2}
\begin{longtable}{|>{\raggedright\arraybackslash}p{2.6cm}|>{\raggedright\arraybackslash}p{3.2cm}|>{\raggedright\arraybackslash}p{5.6cm}|}
\hline
\textbf{Camada} & \textbf{Tecnologias} & \textbf{Justificativa} \\
\hline
\endfirsthead
\hline
\textbf{Camada} & \textbf{Tecnologias} & \textbf{Justificativa} \\
\hline
\endhead
\hline
\endfoot
\hline
\endlastfoot
Frontend & Next.js e React & Interface web responsiva para recepção e dentista com fluxo rápido de agenda. \\
\hline
Backend & Node.js LTS e NestJS & APIs modulares para agenda, prontuário, fechamento de atendimento e integrações externas. \\
\hline
Dados & PostgreSQL e Redis & Persistência transacional e cache para leitura de agenda e disponibilidade. \\
\hline
Infra/Obs. & Docker, CI/CD, OpenTelemetry e Grafana & Operação observável com deploy padronizado e monitoramento contínuo. \\
\hline
\end{longtable}

\section{Modelo de Domínio}
\begin{itemize}
  \item Consulta pertence a um paciente, um profissional e uma clínica.
  \item Prontuário registra evolução por consulta e documentos vinculados.
  \item Atendimento concluído possui status administrativo de fechamento.
\end{itemize}

\section{Requisitos Funcionais}
\subsection{Catálogo de requisitos}
\renewcommand{\arraystretch}{1.2}
\begin{longtable}{|>{\raggedright\arraybackslash}p{1.4cm}|>{\raggedright\arraybackslash}p{6.0cm}|>{\raggedright\arraybackslash}p{2.0cm}|>{\raggedright\arraybackslash}p{2.3cm}|}
\hline
\textbf{ID} & \textbf{Descrição} & \textbf{Prior.} & \textbf{Módulo} \\
\hline
\endfirsthead
\hline
\textbf{ID} & \textbf{Descrição} & \textbf{Prior.} & \textbf{Módulo} \\
\hline
\endhead
\hline
\endfoot
\hline
\endlastfoot
RF-01 & Permitir autoagendamento com validação de disponibilidade por profissional e tipo de atendimento. & Alta & Agenda \\
\hline
RF-02 & Permitir confirmação e lembretes automáticos por WhatsApp/SMS com atualização de status da consulta. & Alta & Agenda \\
\hline
RF-03 & Permitir registro de prontuário odontológico digital com assinatura eletrônica de documentos obrigatórios. & Alta & Prontuário \\
\hline
RF-04 & Permitir fechamento administrativo do atendimento concluído sem depender de confirmações externas. & Alta & Atendimento \\
\hline
RF-05 & Permitir bloqueio de funcionalidades por plano contratado com validação no backend. & Alta & Planos \\
\hline
RF-06 & Permitir remarcação e cancelamento com aplicação de regras de antecedência por clínica. & Média & Agenda \\
\hline
RF-07 & Exibir painel operacional com taxa de ocupação, faltas e confirmações por período. & Média & Gestão \\
\hline
RF-08 & Registrar trilha de auditoria para alterações críticas em agenda, prontuário e fechamento de atendimento. & Alta & Auditoria \\
\hline
\end{longtable}

\section{Requisitos Não Funcionais}
\begin{longtable}{|>{\raggedright\arraybackslash}p{1.4cm}|>{\raggedright\arraybackslash}p{8.0cm}|>{\raggedright\arraybackslash}p{2.0cm}|}
\hline
\textbf{ID} & \textbf{Descrição} & \textbf{Prior.} \\
\hline
\endfirsthead
\hline
\textbf{ID} & \textbf{Descrição} & \textbf{Prior.} \\
\hline
\endhead
\hline
\endfoot
\hline
\endlastfoot
RNF-01 & Disponibilidade mensal mínima de 99,5\% para agenda, prontuário e fechamento de atendimento. & Alta \\
\hline
RNF-02 & Tempo de resposta inferior a 2 segundos no percentil 95 para agenda e prontuário. & Alta \\
\hline
RNF-03 & Segregação lógica de dados por clínica e empresa, sem vazamento entre contextos. & Alta \\
\hline
RNF-04 & Criptografia de dados em trânsito (TLS) e em repouso para dados sensíveis. & Alta \\
\hline
RNF-05 & Auditoria imutável para eventos críticos com retenção mínima de 5 anos. & Alta \\
\hline
RNF-06 & Proteção contra abuso em endpoints públicos de confirmação com rate limit e bloqueio progressivo. & Média \\
\hline
\end{longtable}

\section{Regras de Negócio}
\begin{itemize}
  \item RN-01: Uma consulta só pode ser confirmada se houver vínculo ativo paciente-clínica.
  \item RN-02: Remarcação deve respeitar janela mínima de antecedência por política da clínica.
  \item RN-03: O fechamento administrativo do atendimento deve ocorrer sem validações externas.
  \item RN-04: Assinatura de documento clínico é obrigatória para finalizar determinados tipos de atendimento.
\end{itemize}

\section{Casos de Uso}
\subsection*{CU-01 — Agendar e confirmar consulta}
\textbf{Ator principal:} Paciente\\
\textbf{Fluxo resumido:}
\begin{enumerate}
  \item Paciente seleciona profissional e horário disponível.
  \item Sistema registra consulta e dispara confirmação automática.
  \item Paciente confirma e consulta segue para atendimento.
\end{enumerate}

\subsection*{CU-02 — Registrar atendimento e concluir fechamento}
\textbf{Ator principal:} Dentista/Recepção\\
\textbf{Fluxo resumido:}
\begin{enumerate}
  \item Dentista registra evolução e documentos do atendimento.
  \item Sistema registra o fechamento administrativo do atendimento.
  \item Atendimento é concluído sem bloqueio por etapas financeiras.
\end{enumerate}

\section{Fluxo Principal (Opcional)}
\begin{figure}[htbp]
  \centering
  \MermaidTikzStyles
  \resizebox{\MermaidMaxWidth}{!}{%
    \begin{tikzpicture}[node distance=8mm and 10mm,>=Latex]
      \node[term, small] (a) {Início};
      \node[box, small, below=of a] (b) {Paciente inicia autoagendamento (RF-01)};
      \node[decision, small, below=10mm of b] (c) {Horário disponível?};
      \node[box, small, below left=10mm and 8mm of c] (d) {Exibir alternativas (RF-01)};
      \node[box, small, below right=10mm and 8mm of c] (e) {Registrar consulta (RF-01)};
      \node[box, small, below=of e] (f) {Disparar confirmação automática (RF-02)};
      \node[decision, small, below=10mm of f] (g) {Paciente confirmou?};
      \node[box, small, below left=10mm and 8mm of g] (h) {Remarcar ou cancelar (RF-06)};
      \node[box, small, below right=10mm and 8mm of g] (i) {Atender e registrar prontuário (RF-03)};
      \node[box, small, below=of i] (j) {Registrar fechamento do atendimento (RF-04)};
      \node[term, small, below=12mm of j] (k) {Fim};

      \draw[line] (a) -- (b);
      \draw[line] (b) -- (c);
      \draw[line] (c) -- node[left, small] {Não} (d);
      \draw[line] (c) -- node[right, small] {Sim} (e);
      \draw[line] (e) -- (f);
      \draw[line] (f) -- (g);
      \draw[line] (g) -- node[left, small] {Não} (h);
      \draw[line] (g) -- node[right, small] {Sim} (i);
      \draw[line] (i) -- (j);
      \draw[line] (d) |- (h);
      \draw[line] (h) |- (k);
      \draw[line] (j) -- (k);
    \end{tikzpicture}%
  }
  \caption{Fluxo principal do usuário no Odonto.}
\end{figure}

\section{Critérios de Aceitação}
\subsection*{CA-01 (RF-01)}
\textbf{Dado} um horário disponível na agenda\\
\textbf{Quando} o paciente concluir o autoagendamento\\
\textbf{Então} o sistema deve registrar a consulta sem conflito de disponibilidade.

\subsection*{CA-02 (RF-02)}
\textbf{Dado} uma consulta agendada\\
\textbf{Quando} o horário de confirmação for atingido\\
\textbf{Então} o sistema deve enviar mensagem automática e atualizar o status com base na resposta.

\subsection*{CA-03 (RF-04)}
\textbf{Dado} um atendimento clínico concluído\\
\textbf{Quando} a recepção finalizar o fechamento administrativo\\
\textbf{Então} o sistema deve registrar o fechamento sem exigir validações financeiras externas.

\section{Priorização}
\subsection*{MVP}
\begin{itemize}
  \item RF-01, RF-02, RF-03, RF-04 e RF-05.
  \item RNF-01, RNF-02, RNF-03 e RNF-04.
\end{itemize}

\subsection*{Pós-MVP}
\begin{itemize}
  \item RF-06, RF-07 e RF-08.
  \item RNF-05 e RNF-06.
\end{itemize}

\section{Riscos e Dependências}
\begin{itemize}
  \item Dependência de SLA de provedores de mensagem.
  \item Risco de dados inconsistentes de agenda em clínicas com regras não padronizadas.
  \item Dependência de treinamento operacional para adoção do novo fluxo.
\end{itemize}

\section{Pendências para Refinamento}
\begin{itemize}
  \item Definir política única de antecedência para remarcações por tipo de clínica.
  \item Fechar catálogo oficial de documentos com assinatura obrigatória.
\end{itemize}

\section{Conclusão}
Este documento congela o escopo da versão 1.0 do Odonto com trilha direta entre dor operacional, requisitos e critérios de aceite, habilitando a transição para build.

\end{document}
