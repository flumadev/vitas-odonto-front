\documentclass[12pt,a4paper]{article}

\usepackage[T1]{fontenc}
\usepackage[utf8]{inputenc}
\usepackage[brazil]{babel}
\usepackage{geometry}
\usepackage{enumitem}
\usepackage{hyperref}
\usepackage{graphicx}
\usepackage{tikz}
\usetikzlibrary{arrows.meta,positioning,shapes.geometric}

\geometry{margin=2.5cm}
\setlist[itemize]{leftmargin=1.2cm}
\setlist[enumerate]{leftmargin=1.2cm}

\hypersetup{colorlinks=true,linkcolor=black,urlcolor=blue}

\input{../docs/requirements/mermaid-macros.tex}

\title{Levantamento Lean de Requisitos\\Vitas Odonto}
\author{Produto e Engenharia}
\date{\today}

\begin{document}
\maketitle
\tableofcontents
\newpage

\section{Contexto e Problema}
No Odonto, agenda, confirmação, prontuário e fechamento de atendimento ainda operam com baixa integração de ponta a ponta.
Esse cenário aumenta faltas, eleva retrabalho da recepção e alonga o tempo de conclusão administrativa por atendimento.

\section{Objetivo do MVP}
Validar um fluxo único para atendimento odontológico, do agendamento ao fechamento administrativo, com automação mínima para reduzir atrito operacional.

Metas para até 90 dias após o go-live:
\begin{itemize}
  \item Reduzir em 20\% a taxa de não comparecimento em consultas elegíveis.
  \item Atingir 80\% de confirmações automáticas por WhatsApp/SMS nas consultas elegíveis.
  \item Atingir 90\% de fechamentos administrativos em até 5 minutos após conclusão clínica.
\end{itemize}

\section{Usuários e Stakeholders}
\begin{itemize}
  \item Paciente — precisa agendar e confirmar com rapidez.
  \item Recepção — precisa reduzir confirmações e remarcações manuais.
  \item Dentista — precisa registrar evolução clínica e documentos com segurança jurídica.
  \item Administrativo — precisa reduzir esforço manual no encerramento de atendimentos.
  \item Gestor da clínica — precisa monitorar eficiência operacional e uso por plano.
\end{itemize}

\section{Hipóteses}
\begin{itemize}
  \item Hipótese de valor: fluxo único reduz faltas e retrabalho operacional.
  \item Hipótese de viabilidade: integrações com WhatsApp/SMS e automações de fluxo sustentam o MVP sem congelar arquitetura.
\end{itemize}

\section{Escopo Mínimo}
\subsection*{Incluído}
\begin{itemize}
  \item Autoagendamento 24/7 por profissional e tipo de atendimento.
  \item Confirmação e lembretes automáticos com atualização de status na agenda.
  \item Prontuário odontológico digital com assinatura eletrônica de documentos essenciais.
  \item Fechamento administrativo do atendimento sem dependência de validações externas.
  \item Bloqueio de funcionalidades por plano no backend.
\end{itemize}

\subsection*{Fora de escopo}
\begin{itemize}
  \item Teleconsulta com vídeo.
  \item Financiamento e análise de crédito.
  \item Integração com maquininha de cartão.
  \item Emissão fiscal completa (NFS-e) no MVP.
\end{itemize}

\section{Fluxo Principal (Opcional)}
\begin{figure}[htbp]
  \centering
  \MermaidTikzStyles
  \resizebox{\MermaidMaxWidth}{!}{%
    \begin{tikzpicture}[node distance=8mm and 10mm,>=Latex]
      \node[term, small] (a) {Início};
      \node[box, small, below=of a] (b) {Paciente acessa autoagendamento (RF-01)};
      \node[box, small, below=of b] (c) {Seleciona profissional e horário};
      \node[decision, small, below=10mm of c] (d) {Horário disponível?};
      \node[box, small, below left=10mm and 8mm of d] (e) {Exibir horários alternativos (RF-01)};
      \node[decision, small, below=10mm of e] (f) {Paciente aceita alternativa?};
      \node[box, small, below right=10mm and 8mm of d] (g) {Registrar consulta (RF-01)};
      \node[box, small, below=of g] (h) {Enviar confirmação (RF-02)};
      \node[box, small, below=of h] (i) {Registrar fechamento administrativo (RF-04)};
      \node[box, small, below left=10mm and 8mm of f] (j) {Encerrar sem agendamento};
      \node[term, small, below=12mm of i] (k) {Fim};

      \draw[line] (a) -- (b);
      \draw[line] (b) -- (c);
      \draw[line] (c) -- (d);
      \draw[line] (d) -- node[left, small] {Não} (e);
      \draw[line] (d) -- node[right, small] {Sim} (g);
      \draw[line] (e) -- (f);
      \draw[line] (f) -- node[right, small] {Sim} (g);
      \draw[line] (f) -- node[left, small] {Não} (j);
      \draw[line] (g) -- (h);
      \draw[line] (h) -- (i);
      \draw[line] (i) -- (k);
      \draw[line] (j) |- (k);
    \end{tikzpicture}%
  }
  \caption{Fluxo principal de agendamento, confirmação e fechamento no Odonto.}
\end{figure}

\section{Requisitos Funcionais Essenciais}
\begin{itemize}
  \item RF-01: Permitir autoagendamento com validação de disponibilidade por profissional e tipo de atendimento.
  \item RF-02: Enviar confirmação e lembretes por WhatsApp/SMS com atualização de status do agendamento.
  \item RF-03: Manter prontuário digital com assinatura eletrônica de documentos obrigatórios.
  \item RF-04: Registrar fechamento administrativo do atendimento sem depender de confirmações externas.
  \item RF-05: Aplicar bloqueio de funcionalidades por plano contratado no backend.
\end{itemize}

\section{Requisitos Não Funcionais Críticos}
\begin{itemize}
  \item RNF-01: Disponibilidade mensal mínima de 99,5\% para agenda, prontuário e fechamento de atendimento.
  \item RNF-02: Tempo de resposta menor que 2 segundos no percentil 95 para agenda e prontuário.
  \item RNF-03: Segregação lógica de dados por empresa e clínica, com conformidade LGPD.
\end{itemize}

\section{Riscos e Incertezas}
\begin{itemize}
  \item Instabilidade de provedores de mensagem pode afetar confirmações.
  \item Configurações de agenda heterogêneas podem gerar conflitos de disponibilidade.
  \item Regras administrativas heterogêneas podem impactar o tempo de fechamento por clínica.
\end{itemize}

\section{Critério de Sucesso}
Em até 90 dias após a entrada em produção:
\begin{itemize}
  \item Redução de 20\% na taxa de não comparecimento em consultas elegíveis.
  \item Pelo menos 80\% das confirmações elegíveis executadas automaticamente.
  \item Pelo menos 90\% dos atendimentos com fechamento administrativo em até 5 minutos.
\end{itemize}

\section{Próximos Passos}
\begin{itemize}
  \item Fechar contratos de integração com provedores de mensagem.
  \item Detalhar critérios de aceite de RF-02 e RF-04.
  \item Preparar PRD completo e levantamento técnico para \texttt{Ready for Build}.
\end{itemize}

\end{document}
