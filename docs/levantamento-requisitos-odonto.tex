\documentclass[12pt,a4paper]{article}

\usepackage[T1]{fontenc}
\usepackage[utf8]{inputenc}
\usepackage[brazil]{babel}
\usepackage{geometry}
\usepackage{enumitem}
\usepackage{hyperref}
\usepackage{graphicx}
\usepackage{tikz}
\usetikzlibrary{arrows.meta,positioning,shapes.geometric}

\geometry{margin=2.5cm}
\setlist[itemize]{leftmargin=1.2cm}
\setlist[enumerate]{leftmargin=1.2cm}

\hypersetup{
  colorlinks=true,
  linkcolor=black,
  urlcolor=blue
}

\input{../docs/requirements/mermaid-macros.tex}

\title{Levantamento Lean de Requisitos\\Vitas Odonto}
\author{Produto e Engenharia}
\date{\today}

\makeatletter
\renewcommand{\maketitle}{
  \begin{titlepage}
    \centering
    \vspace*{\fill}
    {\LARGE\bfseries \@title\par}
    \vspace{1.2cm}
    {\large \@author\par}
    \vspace{0.8cm}
    {\large \@date\par}
    \vspace*{\fill}
  \end{titlepage}
}
\makeatother

\begin{document}
\hypersetup{pageanchor=false}
\maketitle
\hypersetup{pageanchor=true}
\tableofcontents
\newpage

\section{Contexto e Problema}
Hoje, o módulo Odonto opera com etapas críticas em ferramentas separadas (agenda, confirmação, prontuário e cobrança), sem um fluxo único de ponta a ponta.
Sem essa integração, a clínica perde previsibilidade operacional: há mais faltas, mais esforço manual da recepção e maior tempo até o fechamento financeiro.

\section{Objetivo do MVP}
Validar um fluxo único e operacional para o atendimento odontológico, cobrindo agendamento, confirmação e cobrança no mesmo processo.

Metas mensuráveis para até 90 dias após o go-live:
\begin{itemize}
  \item Reduzir em 20\% a taxa de não comparecimento em consultas confirmáveis.
  \item Atingir pelo menos 80\% de confirmações automáticas em consultas elegíveis (WhatsApp ou SMS).
  \item Atingir pelo menos 90\% de baixas automáticas de Pix em até 5 minutos após confirmação do pagamento.
\end{itemize}

\section{Usuários e Stakeholders}
\begin{itemize}
  \item Paciente: precisa agendar e confirmar consulta com rapidez e clareza.
  \item Recepção: precisa reduzir confirmação e remarcação manual.
  \item Dentista: precisa manter evolução clínica e documentos com segurança jurídica.
  \item Financeiro: precisa reduzir conciliação manual de cobrança.
  \item Gestor da clínica: precisa controlar eficiência operacional e aderência por plano.
\end{itemize}

\section{Hipóteses}
\begin{itemize}
  \item Hipótese de valor: um fluxo único reduz faltas e retrabalho da operação.
  \item Hipótese de viabilidade: integração com WhatsApp/SMS e Pix é suficiente para validar o MVP sem congelar arquitetura.
\end{itemize}

\section{Escopo Mínimo}
\subsection*{Incluído}
\begin{itemize}
  \item Autoagendamento 24/7 por profissional e tipo de atendimento.
  \item Confirmação e lembrete automáticos com atualização de status na agenda.
  \item Prontuário digital odontológico com assinatura eletrônica de documentos essenciais.
  \item Cobrança Pix com baixa automática após confirmação de pagamento.
  \item Bloqueio de funcionalidades por plano contratado no backend.
\end{itemize}

\subsection*{Fora de escopo}
\begin{itemize}
  \item Teleconsulta com vídeo em tempo real.
  \item Financiamento de tratamento e análise de crédito.
  \item Integração com maquininha de cartão.
  \item Emissão fiscal completa (NFS-e) neste MVP.
\end{itemize}

\section{Fluxo Principal (Opcional)}
\begin{figure}[htbp]
  \centering
  \MermaidTikzStyles
  \resizebox{\MermaidMaxWidth}{!}{%
    \begin{tikzpicture}[
      node distance=8mm and 10mm,
      >=Latex
    ]
      \node[term, small] (a) {Início};
      \node[box, small, below=of a] (b) {Paciente acessa autoagendamento (RF-01)};
      \node[box, small, below=of b] (c) {Seleciona profissional e horário};
      \node[box, small, below=of c] (d) {Sistema valida disponibilidade e regras (RF-01)};
      \node[decision, small, below=10mm of d] (e) {Horário disponível?};
      \node[box, small, below left=11mm and 8mm of e] (f) {Exibir horários alternativos (RF-01)};
      \node[decision, small, below=11mm of f] (g) {Paciente aceita alternativa?};
      \node[box, small, below right=11mm and 8mm of e] (h) {Registrar consulta na agenda (RF-01)};
      \node[box, small, below=of h] (i) {Enviar confirmação automática (RF-02)};
      \node[box, small, below=of i] (j) {Atualizar status para confirmado (RF-02)};
      \node[box, small, below=of j] (k) {Gerar cobrança Pix e baixar automaticamente (RF-04)};
      \node[box, small, below left=11mm and 8mm of g] (l) {Encerrar tentativa sem agendamento};
      \node[term, small, below=13mm of k] (m) {Fim};

      \draw[line] (a) -- (b);
      \draw[line] (b) -- (c);
      \draw[line] (c) -- (d);
      \draw[line] (d) -- (e);
      \draw[line] (e) -- node[left, small] {Não} (f);
      \draw[line] (e) -- node[right, small] {Sim} (h);
      \draw[line] (f) -- (g);
      \draw[line] (g) -- node[right, small] {Sim} (h);
      \draw[line] (g) -- node[left, small] {Não} (l);
      \draw[line] (h) -- (i);
      \draw[line] (i) -- (j);
      \draw[line] (j) -- (k);
      \draw[line] (k) -- (m);
      \draw[line] (l) |- (m);
    \end{tikzpicture}%
  }
  \caption{Fluxo principal do autoagendamento até confirmação e cobrança no Odonto.}
\end{figure}

\section{Requisitos Funcionais Essenciais}
\begin{itemize}
  \item RF-01: O sistema deve permitir autoagendamento 24/7 com validação de disponibilidade por profissional e tipo de atendimento.
  \item RF-02: O sistema deve enviar confirmação e lembretes automáticos por WhatsApp ou SMS e atualizar o status do agendamento conforme resposta do paciente.
  \item RF-03: O sistema deve manter prontuário digital com evolução clínica e assinatura eletrônica de documentos obrigatórios.
  \item RF-04: O sistema deve gerar cobrança Pix com QR Code/link e realizar baixa automática após confirmação do pagamento.
  \item RF-05: O sistema deve aplicar bloqueio de funcionalidades por plano contratado no backend.
\end{itemize}

\section{Requisitos Não Funcionais Críticos}
\begin{itemize}
  \item RNF-01: Disponibilidade mensal mínima de 99,5\% para agenda, prontuário e cobrança.
  \item RNF-02: Tempo de resposta menor que 2 segundos no percentil 95 para agenda e prontuário.
  \item RNF-03: Segregação lógica de dados por empresa e clínica, com conformidade LGPD.
\end{itemize}

\section{Riscos e Incertezas}
\begin{itemize}
  \item Instabilidade de provedores de WhatsApp/SMS pode reduzir taxa de confirmação.
  \item Divergências de configuração de agenda entre clínicas podem gerar conflitos de disponibilidade.
  \item Falhas de integração com PSP podem afetar baixa automática do Pix.
  \item Configuração incorreta de planos pode liberar funcionalidades indevidas.
\end{itemize}

\section{Critério de Sucesso}
Em até 90 dias após a entrada em produção:
\begin{itemize}
  \item Redução de 20\% na taxa de não comparecimento em consultas elegíveis.
  \item Pelo menos 80\% de confirmações elegíveis executadas de forma automática.
  \item Pelo menos 90\% das cobranças Pix com baixa automática em até 5 minutos.
\end{itemize}

\section{Próximos Passos}
\begin{itemize}
  \item Fechar contratos de integração com provedores de WhatsApp/SMS e PSP de Pix.
  \item Detalhar critérios de aceite de RF-02 e RF-04 com Produto, Engenharia e Financeiro.
  \item Preparar levantamento técnico para etapa \texttt{Ready for Build}.
\end{itemize}

\end{document}
